\documentclass[a4paper]{article}
\usepackage{geometry}
\usepackage{graphicx}
\usepackage{epsfig}
\usepackage{amsmath}
\usepackage{indentfirst}
\usepackage{float}
\usepackage{setspace}
\usepackage{amsfonts}
\usepackage{hyperref}
\usepackage{booktabs}
\usepackage{caption}
\usepackage{subfigure}
\usepackage{times}
\usepackage{hyperref}
\usepackage[round]{natbib}
\usepackage{verbatim}
\usepackage{colortbl}
\usepackage{lscape}
\usepackage[affil-it]{authblk}


\geometry{left=2cm,right=2cm,top=2cm,bottom=2cm}

\setlength{\parindent}{2em}

\begin{document}
	
	\title{Narrative Conservatism}
	
	\author{Beatriz Garc\'ia Osma}
	
	\author{Juan Manuel Garc\'ia Lara}
	
	\author{Fengzhi Zhu%
		\thanks{Email: \texttt{fzhu@emp.uc3m.es}}}
	
	\affil{Department of Business Administration, Universidad Carlos III de Madrid, Spain}
	
	\date{Dated: \today}
	
	\maketitle

\begin{spacing}{2}
\begin{abstract}
	I study whether firms' level of corporate social responsibility (CSR) affects their speed of response to gains and losses in financial reports, i.e. conditional conservatism. Using a natural experiment of staggered constituency statute enactments in the U.S. during 1980s, which allow managers to consider stakeholder interest in decision-making and thus improves CSR overall, I find that conservatism increases after CSR improvement. Such increase in conservatism is more significant in firms with strong debt-contracting demand and high managerial ability, consistent with debt being the main resource of conservatism and managerial style playing a role in determining CSR and conservatism. The results are robust to various alternative sample selections, variable measurements and model specifications. This study contributes to the linkage between CSR and conservatism literature by using a unique setting to document a positive causal impact of CSR on conservatism. Furthermore, this paper adds to the discussion on social influence of constituency statutes by providing a novel accounting viewpoint.\\
	\newline
	
	\textbf{Keywords}: \textit{corporate social responsibility;  conditional conservatism; constituency statutes; trust}
\end{abstract}

\clearpage

\section{Introduction}
\cite{}
%[motivation]

%[setting]

%[data]

%[robustness]

%[contribution]

The rest of the paper structures as follows: Section 2 describes theoretical framework. Section 3 explains empirical models and data construction. Section 4 presents main results. Section 5 performs robustness checks. Section 6 concludes. \cite{liTextualAnalysisCorporate2010}

\section{Theoretical Framework}

\begin{center}
	H1: Firms' conditional conservatism decreases as CSR increases.
\end{center}

\begin{center}
	H2: Firms' conditional conservatism increases as CSR increases.
\end{center}

\section{Research Design}
\subsection{News proxy: Market Returns}

\subsection{Text properties}

 \begin{equation}
 	EARN_{i,t}=\alpha_i+\omega_t+\beta_1RET_{i,t}+\beta_2NEG_{i,t}+\beta_3RET_{i,t}\times NEG_{i,t}+\epsilon_{i,t}
 \end{equation}

\subsection{Data}

\section{Results}

\subsection{Summary Statistics}

\subsection{Main Results}

\section{Robustness Checks}

\subsection{Other news proxy}

\section{Conclusions}

\end{spacing}

\newpage
\section{Appendix}
\subsection{Appendix A: 10-Q and 8-K scraping}

\subsection{Appendix B: Financial Variable Definition}
\begin{table}[H]
	\centering
	\begin{tabular}{lp{15cm}p{15cm}}
		\textbf{Variable} & \textbf{Definition} \\
		BTM   & Book to market ratio: book value of equity (Compustat CEQ) divided by market value of equity (Compustat PRCC\_F $\times$ Compustat CSHO), as of current fiscal year. \\
		CASH  & Cash holdings: cash and short-term investments (Compustat CHE) to the book value of total assets (Compustat AT), as of current fiscal year. \\
		DCD   & Debt-contracting demand defined in two ways. (1) DCD1: a dummy variable that takes 1 if a firm experience an increase in average leverage ratio between the pre-enactment and post-enactment period (LEV\_POST $>$ LEV\_PRE), and 0 otherwise. (2) DCD2: The difference between average leverage ratios of the pre-enactment and post-enactment period (LEV\_POST - LEV\_PRE). \\
		EARN  & Earnings: income before extraordinary items (Compustat IB) divided by lagged market value of equity (Compustat PRCC\_F $\times$ Compustat CSHO). \\
		LEV   & Leverage ratio: short term debt (Compustat DLC) plus long term debt (Compustat DLTT) divided by market value of equity, as of current fiscal year. \\
		MA    & Managerial ability: an index of managerial ability constructed by \cite{demerjianQuantifyingManagerialAbility2012}. \\
		MTB   & Market to book ratio: inverse of BTM. \\
		NEG   & Dummy variable for bad news, which takes 1 when market -adjusted stock return (RET) is negative and is 0 otherwise. \\
		POST  & Dummy variable for firm-year observation subject to constituency statute, which takes 1 when a firm-year observation appears after the year in which the constituency statute is enacted in the firm's state of incorporation, and is 0 otherwise. \\
		RET   & Adjusted market return: sum of monthly buy-and-hold stock return (CRSP RET) over the fiscal year (starting from the fourth month of the fiscal year) minus the sum of monthly value-weighted stock return (CRSP VWRETD) over the same period.  \\
		ROA   & Return on assets: operating income before depreciation (Compustat OIBDP) divided by total assets (Compustat AT), as of current fiscal year. \\
		SIZE  & Firm size: natural log of market value of equity (Compustat PRCC\_F $\times$ Compustat CSHO), as of current fiscal year. \\
		TOBINSQ & Tobins' q: market value of total assets, which equals to book value of total assets (Compustat AT) plus market value of equity (Compustat PRCC\_F $\times$ Compustat CSHO) minus the sum of the book value of common stock (Compustat CEQ) and balance sheet deferred taxes (Compustat TXDB), divided by book value of total assets (Compustat AT), as of current fiscal year. \\
	\end{tabular}%
\end{table}%

\subsection{Appendix C: Text Variable Definition}
\begin{table}[H]
	\centering
	\begin{tabular}{lp{15cm}p{15cm}}
		\textbf{Indexes} & \textbf{Definition} \\
		KLD\_TOTAL &  Total number of strengths minus total number of concerns (KLD\_STR - KLD\_CON) \\
		KLD\_STR &  Number of strengths across all five dimensions (env\_str + com\_str + pro\_str + emp\_str + hum\_str) \\
		KLD\_CON &  Number of concerns across all five dimensions (env\_con + com\_con + pro\_con + emp\_con + hum\_con) \\
		com\_str  &  Total number of community strengths, including 6 subcategories: Charitable Giving, Innovative Giving, Non-US Charitable Giving, Support for Housing, Support for Education, and Other Strength. \\
		com\_con  &  Total number of community concerns, including 4 subcategories: Investment Controversies, Negative Economic, Tax Disputes, and Other Concern. \\
		div\_str  &  Total number of diversity strengths, including 6 subcategories: Promotion, Work/Life Benefits, Women \& Minority Contracting, Employment of the Disabled, Gay \& Lesbian Policies, and Other Strength. \\
		div\_con  &  Total number of diversity concerns, including 2 subcategories: Controversies, and Other Concern. \\
		env\_str  &  Total number of environment strengths, including 5 subcategories: Beneficial Products and Services, Pollution Prevention, Recycling, Clean Energy, and Other Strength. \\
		env\_con  &  Total number of environment concerns, including 7 subcategories: Hazardous Waste, Regulatory Problems, Ozone Depleting Chemicals, Substantial Emissions, Agricultural Chemicals, Climate Change, and Other Concern. \\
		emp\_str  &  Total number of employee relations strengths, including 6 subcategories: Union Relations, Cash Profit Sharing, Employee Involvement, Retirement Benefits Strength, Health and Safety Strength, and Other Strength. \\
		emp\_con  &  Total number of employee relations concerns, including 4 subcategories: Union Relations, Health and Safety Concern, Work force Reductions and Retirement Benefits Concern. \\
		hum\_str  &  Total number of human rights strengths, including 2 subcategories: Indigenous Peoples Relations and Labor Rights. \\
		hum\_con  &  Total number of human rights concerns, including 4 subcategories: Burma Concern, Labor Rights, Indigenous Peoples Relations, and Other Concern. \\
		pro\_str  &  Total number of product strengths, including 4 subcategories: Quality, R\&D/Innovation, Benefits to Economically Disadvantaged, and Other \\
		pro\_con  &  Total number of product concerns, including 4 subcategories: Product Safety, Marketing/Contracting, Antitrust, and Other Concern. \\
	\end{tabular}%
\end{table}%

\subsection{Online Appendix}
Tables of untabulated results can be accessed via this website:

%\href{https://drive.google.com/file/d/1P8XRZRd68AaeMenJVAEf1wEs_eRkLUfu/view?usp=sharing}{https://drive.google.com/file/d/1P8XRZRd68AaeMenJVAEf1wEs_eRkLUfu/view?usp=sharing}.

\newpage
\bibliographystyle{plainnat}
\bibliography{NC}

\newpage
%%%%%%%%%%%%%%%%%%%%%%%%%TABLE 1
\begin{table}[H]
	\centering
	\caption{Constituency Statute' Enactment Years (1975 - 2013)}
	\begin{tabular}{lrrrr}
		\toprule
		\toprule
		State & \multicolumn{1}{l}{Enactment date} & \multicolumn{1}{l}{Before} & \multicolumn{1}{l}{After} & \multicolumn{1}{l}{Total} \\
		\midrule
		AZ    & \multicolumn{1}{l}{07/22/1987} & 57    & 89    & 146 \\
		CT    & \multicolumn{1}{l}{06/07/1988} & 207   & 307   & 514 \\
		FL    & \multicolumn{1}{l}{06/27/1989} & 443   & 756   & 1199 \\
		GA    & \multicolumn{1}{l}{07/01/1989} & 292   & 518   & 810 \\
		HI    & \multicolumn{1}{l}{06/07/1989} & 39    & 52    & 91 \\
		IA    & \multicolumn{1}{l}{12/31/1989} & 95    & 130   & 225 \\
		ID    & \multicolumn{1}{l}{03/22/1988} & 13    & 26    & 39 \\
		IL    & \multicolumn{1}{l}{08/23/1985} & 139   & 277   & 416 \\
		IN    & \multicolumn{1}{l}{04/01/1986} & 261   & 519   & 780 \\
		KY    & \multicolumn{1}{l}{07/15/1988} & 49    & 74    & 123 \\
		LA    & \multicolumn{1}{l}{07/10/1988} & 77    & 114   & 191 \\
		MA    & \multicolumn{1}{l}{07/18/1989} & 687   & 969   & 1656 \\
		MD    & \multicolumn{1}{l}{06/01/1999} & 432   & 352   & 784 \\
		ME    & \multicolumn{1}{l}{09/19/1985} & 74    & 101   & 175 \\
		MN    & \multicolumn{1}{l}{06/01/1987} & 472   & 1,013 & 1485 \\
		MO    & \multicolumn{1}{l}{05/06/1986} & 165   & 306   & 471 \\
		MS    & \multicolumn{1}{l}{07/01/1990} & 2     & 23    & 25 \\
		NC    & \multicolumn{1}{l}{10/01/1993} & 262   & 250   & 512 \\
		NE    & \multicolumn{1}{l}{04/08/1988} & 50    & 32    & 82 \\
		NJ    & \multicolumn{1}{l}{02/04/1989} & 597   & 809   & 1406 \\
		NM    & \multicolumn{1}{l}{04/09/1987} & 30    & 47    & 77 \\
		NV    & \multicolumn{1}{l}{10/01/1991} & 419   & 564   & 983 \\
		NY    & \multicolumn{1}{l}{07/23/1987} & 1,351 & 1,999 & 3350 \\
		OH    & \multicolumn{1}{l}{10/10/1984} & 574   & 1,351 & 1925 \\
		OR    & \multicolumn{1}{l}{03/05/1989} & 134   & 223   & 357 \\
		PA    & \multicolumn{1}{l}{04/27/1990} & 883   & 1,144 & 2027 \\
		RI    & \multicolumn{1}{l}{07/03/1990} & 66    & 110   & 176 \\
		SD    & \multicolumn{1}{l}{07/01/1990} & 27    & 45    & 72 \\
		TN    & \multicolumn{1}{l}{03/11/1988} & 105   & 219   & 324 \\
		TX    & \multicolumn{1}{l}{01/01/2006} & 594   & 241   & 835 \\
		VA    & \multicolumn{1}{l}{03/31/1988} & 341   & 594   & 935 \\
		VT    & \multicolumn{1}{l}{04/16/1998} & 58    & 28    & 86 \\
		WI    & \multicolumn{1}{l}{06/13/1987} & 299   & 559   & 858 \\
		WY    & \multicolumn{1}{l}{01/01/1990} & 32    & 52    & 84 \\
		\midrule
		Total  &       & 9,326 & 13,893 & 23,219 \\
		\bottomrule
		\bottomrule
	\end{tabular}%
	\label{tab:addlabel}%
\end{table}%
\noindent This table presents the enactment dates and number of observations before and after enactment for 34 states out of 35 states that have adopted constituency statutes. North Dakota is excluded because of missing observations before or after law enactment date. Nebraska enacted constituency statute from 1988 to 1995 and from 2007 untill present.

%%%%%%%%%%%%%%%%%%%%%%%%%TABLE 2
\begin{table}[H]
	\centering
	\caption{Basu Summary Statistics (1975-2013)}
	Panel A. Firm-year Observations Before Constituency Statute Law Enactment (N=9,326)
	\begin{tabular}{lrrrrrrrrr}
		\toprule
		\toprule
		& \multicolumn{1}{l}{mean} & \multicolumn{1}{l}{median} & \multicolumn{1}{l}{std. dev.} & \multicolumn{1}{l}{max} & \multicolumn{1}{l}{min} & \multicolumn{1}{l}{p1} & \multicolumn{1}{l}{p25} & \multicolumn{1}{l}{p75} & \multicolumn{1}{l}{p99} \\
		\midrule
		EARN  & 0.088 & 0.098 & 0.151 & 0.378 & -1.034 & -0.524 & 0.049 & 0.163 & 0.378 \\
		RET   & 0.032 & 0.023 & 0.338 & 1.419 & -1.194 & -0.817 & -0.156 & 0.194 & 1.086 \\
		NEG   & 0.466 & 0.000 & 0.499 & 1.000 & 0.000 & 0.000 & 0.000 & 1.000 & 1.000 \\
		SIZE  & 4.310 & 4.144 & 1.953 & 10.388 & 0.623 & 0.623 & 2.841 & 5.636 & 9.264 \\
		BTM   & 0.882 & 0.773 & 0.576 & 3.256 & -0.980 & 0.054 & 0.481 & 1.153 & 3.010 \\
		LEV   & 0.692 & 0.348 & 0.972 & 7.773 & 0.000 & 0.000 & 0.093 & 0.934 & 5.024 \\
		\bottomrule
		\bottomrule
	\end{tabular}%
	\bigbreak
	Panel B. Firm-year Observations After Constituency Statutes Law Enactment (N=13,893)
	\begin{tabular}{lrrrrrrrrr}
		\toprule
		\toprule
		& \multicolumn{1}{l}{mean} & \multicolumn{1}{l}{median} & \multicolumn{1}{l}{std. dev} & \multicolumn{1}{l}{max} & \multicolumn{1}{l}{min} & \multicolumn{1}{l}{p1} & \multicolumn{1}{l}{p25} & \multicolumn{1}{l}{p75} & \multicolumn{1}{l}{p99} \\
		\midrule
		EARN  & 0.025 & 0.060 & 0.162 & 0.378 & -1.034 & -0.820 & 0.023 & 0.086 & 0.290 \\
		RET   & 0.013 & -0.005 & 0.351 & 1.419 & -1.194 & -0.924 & -0.173 & 0.180 & 1.156 \\
		NEG   & 0.509 & 1.000 & 0.500 & 1.000 & 0.000 & 0.000 & 0.000 & 1.000 & 1.000 \\
		SIZE  & 5.589 & 5.628 & 2.261 & 10.388 & 0.623 & 0.790 & 3.875 & 7.245 & 10.388 \\
		BTM   & 0.682 & 0.574 & 0.548 & 3.256 & -0.980 & -0.581 & 0.359 & 0.851 & 2.909 \\
		LEV   & 0.568 & 0.260 & 0.953 & 7.773 & 0.000 & 0.000 & 0.068 & 0.685 & 5.293 \\
		\bottomrule
		\bottomrule
	\end{tabular}%
	\bigbreak
	Panel C. Variable Difference Before and After Contituency Statute Law Enactment
	\begin{tabular}{lcc}
		\toprule
		\toprule
		& \multicolumn{1}{l}{mean difference} & \multicolumn{1}{l}{t-statistic} \\
		\midrule
		EARN  & -0.06*** & -29.76 \\
		RET   & -0.02*** & -4.18 \\
		NEG   & 0.04*** & 6.47 \\
		SIZE  & 1.28*** & 44.61 \\
		BTM   & -0.20*** & -26.79 \\
		LEV   & -0.12*** & -9.60 \\
		\bottomrule
		\bottomrule
	\end{tabular}%
\end{table}%
\noindent This table reports summary statistics for key variables used in Basu measure, separated into two time periods: before (Panel A) and after (Panel B) constituency statute enactment. Panel C demonstrates the differences in mean value between pre-enactment and post-enactment period for all key variables, and the significance of mean differences. EARN, RET, SIZE, BTM and LEV are winsorized at 1 and 99 percentiles. All variables are defined in Appendix B. *, **, and *** indicate significance at 10\%, 5\% and 1\% confidence level respectively.

%%%%%%%%%%%%%%%%%%%%%%%%%TABLE 3
\newgeometry{left=1cm, right=1cm, top=1cm, bottom=1.8cm}
\begin{landscape}
	\begin{table}[H]
		\centering
		\caption{Effect of Constituency Statue Enactments on Conservatism, Three Periods}
		$EARN_{i,t}=\alpha_i+\omega_t+\beta_1RET_{i,t}+\beta_2NEG_{i,t}+\beta_3RET_{i,t}\times NEG_{i,t}+\epsilon_{i,t}$ \qquad \qquad \qquad \qquad \qquad \qquad \qquad \qquad \qquad \qquad \qquad \qquad \qquad \qquad (1)
		
		$EARN_{i,t}=\alpha_i+\omega_t+\beta_1RET_{i,t}+\beta_2NEG_{i,t}+\beta_3RET_{i,t}\times NEG_{i,t}+POST_{i,t}\times (\beta_4+\beta_5RET_{i,t}+\beta_6NEG_{i,t}+\beta_7RET_{i,t}\times NEG_{i,t})+\epsilon_{i,t}$ (2)
		
		$\beta_j=\sum_{K}^{}\delta_{j,k}State_k+\sum_{M}^{}\theta_{j,m}Year_m \qquad j=1,2,3$ \qquad \qquad \qquad \qquad \qquad \qquad \qquad \qquad \qquad \qquad \qquad \qquad \qquad \qquad \qquad \qquad \qquad \quad (3)
		
		\begin{tabular}{lccccccc}
			\toprule
			\toprule
			Dependent Variable: EARN & Baseline Basu & \multicolumn{2}{c}{DiD: 1975-2013} & \multicolumn{2}{c}{1984-1992} & \multicolumn{2}{c}{1985-1990} \\
			& I     & II & III & IV & V& VI & VII \\
			&  &  & Incorp. = State &  & Incorp. = State&  & Incorp. = State\\
			\midrule
			RET   & 0.0603*** &       &       &       &       &       &  \\
			& 6.82  &       &       &       &       &       &  \\
			NEG   & 0.0019 &       &       &       &       &       &  \\
			& 0.70  &       &       &       &       &       &  \\
			\rowcolor[rgb]{ .851,  .851,  .851} RET $\times$ NEG & 0.1171*** &       &       &       &       &       &  \\
			\rowcolor[rgb]{ .851,  .851,  .851}       & 10.07 &       &       &       &       &       &  \\
			POST  &       & -0.006 & 0.0067 & -0.0204 & -0.0029 & -0.0098 & 0.0213 \\
			&       & -0.93 & 0.87  & -1.46 & -0.15 & -0.68 & 1.06 \\
			\rowcolor[rgb]{ .851,  .851,  .851} POST $\times$ RET &       & -0.0114 & -0.0334 & 0.0896 & 0.0508 & 0.0762 & -0.0312 \\
			\rowcolor[rgb]{ .851,  .851,  .851}       &       & -0.37 & -1.07 & 1.18  & 0.49  & 0.87  & -0.26 \\
			POST $\times$ NEG &       & 0.0035 & 0.0091 & -0.0012 & -0.0092 & -0.0133 & -0.0377 \\
			&       & 0.35  & 0.90  & -0.07 & -0.45 & -0.47 & -1.38 \\
			\rowcolor[rgb]{ .851,  .851,  .851} POST $\times$ RET $\times$ NEG &       & 0.1041* & 0.1961*** & 0.0109 & 0.0055 & 0.0421 & 0.0833 \\
			\rowcolor[rgb]{ .851,  .851,  .851}       &       & 2.05  & 4.37  & 0.09  & 0.04  & 0.36  & 0.60 \\
			Constant & 0.1597*** & 0.0644** & 0.0571** & 0.0277 & 0.0171 & 0.0442** & 0.0127 \\
			& 17.27 & 3.45  & 2.77  & 1.53  & 0.87  & 2.80 & 0.64 \\
			Year Fixed Effects (Main) & YES   & YES   & YES   & YES   & YES   & YES   & YES \\
			Firm Fixed Effects (Main) & YES   & YES   & YES   & YES   & YES   & YES   & YES \\
			Year Fixed Effects (Basu Coefficients) & NO    & YES   & YES   & YES   & YES   & YES   & YES \\
			State Fixed Effects (Basu Coefficients) & NO    & YES   & YES   & YES   & YES   & YES   & YES \\
			S.E. Clustered by States & YES   & YES   & YES   & YES   & YES   & YES   & YES \\
			Observations  & 23,219 &                  23,219  &                  17,347  &                     7,202  &                     5,314  &                     4,806  &                    3,514  \\
			Adj. R-square & 0.2970 & 0.3108 & 0.3229 & 0.3645 & 0.3716 & 0.3916 & 0.3950 \\
			\bottomrule
			\bottomrule
		\end{tabular}%
	\end{table}%
	
	\noindent This table reports results of baseline Basu measure (Equation 1) over full sample period (Column I), and DiD results (Equation 2 and 3) in three sample periods: 1975 - 2013 (Column II, III), 1984 - 1992 (Column IV, V) and 1985 - 1990 (Column VI, VII). Column III, Column V and Column VII show the DiD results using only firms that headquarter in their state of incorporation. All variables are defined in Appendix B. EARN and RET are winsorized at 1 and 99 percentiles. All regressions control for firm and year fixed effects. DiD regressions control for Basu coefficient fixed effects. Standard errors are clustered at state of incorporation level. %The Basu coefficient fixed effects absorb slope coefficients in baseline Basu model so they are omitted.  *, **, and *** indicate significance at 10\%, 5\% and 1\% confidence level respectively. T-statistics are reported below coefficients.
	
\end{landscape}
\restoregeometry

%\begin{table}[H]
%\centering
%\caption{Basu Measure of Conditional Conservatism}
%$EARN_{i,t}=\alpha_i+\omega_t+\beta_1RET_{i,t}+\beta_2NEG_{i,t}+\beta_3RET_{i,t}\times NEG_{i,t}+\epsilon_{i,t}$
%\begin{tabular}{lccc}
%\toprule
%\toprule
%Dep. Var. EARN & Prediction & 1975-2013 & 1985-1990 \\
%\midrule
%Constant &       & 0.1597*** & 0.0681*** \\
%&       & 17.27 & 12.54 \\
%RET   &       & 0.0603*** & 0.0793** \\
%&       & 6.82  & 2.84 \\
%NEG   &       & 0.0019 & 0.0087 \\
%&       & 0.70  & 1.21 \\
%\rowcolor[rgb]{ .851,  .851,  .851} RET $\times$ NEG & (+)   & 0.1171*** & 0.0932* \\
%\rowcolor[rgb]{ .851,  .851,  .851}       &       & 10.07 & 2.71 \\
%Year Fixed Effects  &       & YES   & YES \\
%Firm Fixed Effects &       & YES   & YES \\
%S.E. Clustered by States &       & YES   & YES \\
%Observations     &       & 23,219 & 4,806 \\
%Adj. R-square &       & 0.2970 & 0.3781 \\
%\bottomrule
%\bottomrule
%\end{tabular}%
%\end{table}%
%
%%%%%%%%%%%%%%%%TABLE 4
%\begin{table}[H]
%\centering
%\caption{DiD: Effect of Constituency Statue Enactments on Conservatism, Two Periods}
%
%$EARN_{i,t}=\alpha_i+\omega_t+\beta_1RET_{i,t}+\beta_2NEG_{i,t}+\beta_3RET_{i,t}\times NEG_{i,t}+POST_{i,t}\times (\beta_4+\beta_5RET_{i,t}+\beta_6NEG_{i,t}+\beta_7RET_{i,t}\times NEG_{i,t})+\epsilon_{i,t}$
%
%\begin{tabular}{lcccc}
%\toprule
%\toprule
%Dep. Var. EARN & \multicolumn{2}{c}{1975-2013} & \multicolumn{2}{c}{1985-1990} \\
%& I & II& III & IV\\
%& Full & Incorp.=State & Full & Incorp.=State\\
%\midrule
%POST  & -0.006 & 0.0067 & -0.0098 & 0.0213 \\
%& -0.93 & 0.87  & -0.68 & 1.06 \\
%\rowcolor[rgb]{ .851,  .851,  .851} POST $\times$ RET & -0.0114 & -0.0334 & 0.0762 & -0.0312 \\
%\rowcolor[rgb]{ .851,  .851,  .851}       & -0.37 & -1.07 & 0.87  & -0.26 \\
%POST $\times$ NEG & 0.0035 & 0.0091 & -0.0133 & -0.0377 \\
%& 0.35  & 0.90  & -0.47 & -1.38 \\
%\rowcolor[rgb]{ .851,  .851,  .851} POST $\times$ RET $\times$ NEG & 0.1041* & 0.1961*** & 0.0421 & 0.0833 \\
%\rowcolor[rgb]{ .851,  .851,  .851}       & 2.05  & 4.37  & 0.36  & 0.60 \\
%Constant & 0.0644** & 0.0571** & 0.0442** & 0.0127 \\
%& 3.45  & 2.77  & 2.80   & 0.64 \\
%Year Fixed Effects (Main) & YES   & YES   & YES   & YES \\
%Firm Fixed Effects (Main) & YES   & YES   & YES   & YES \\
%Year Fixed Effects (Basu Coefficients) & YES   & YES   & YES   & YES \\
%State Fixed Effects (Basu Coefficients) & YES   & YES   & YES   & YES \\
%S.E. Clustered by States & YES   & YES   & YES   & YES \\
%Observations  &                  23,219  &                  17,347  &                     4,806  &                     3,514  \\
%Adj. R-square & 0.3108 & 0.3229 & 0.3916 & 0.3950 \\
%\bottomrule
%\bottomrule
%\end{tabular}%
%\end{table}%

%%%%%%%%%%TABLE 4
\begin{landscape}
	\begin{table}[H]
		\centering
		\caption{DiD: Effect of Constituency Statue Enactments on Conservatism, Rolling Windows}
		$EARN_{i,t}=\alpha_i+\omega_t+\beta_1RET_{i,t}+\beta_2NEG_{i,t}+\beta_3RET_{i,t}\times NEG_{i,t}+Dummy_{i,t}\times (\beta_4+\beta_5RET_{i,t}+\beta_6NEG_{i,t}+\beta_7RET_{i,t}\times NEG_{i,t})+\epsilon_{i,t}$
		\begin{tabular}{lcccccc}
			\toprule
			\toprule
			Dep. Var. EARN & I & II & III & IV & V & VI \\
			& -1/+1 &  -1/+1 & -2/+2 & -2/+2 & -3/+3 & -3/+3 \\
			& Full & Incorp.= State & Full & Incorp.= State & Full & Incorp.= State \\
			\midrule
			\rowcolor[rgb]{ .906,  .902,  .902} Dummy $\times$ RET$\times$ NEG & 0.1717 & 0.1257 & 0.1622* & 0.1291 & 0.1358** & 0.1266* \\
			& 2.02  & 1.26  & 2.24  & 1.65  & 2.93  & 2.71 \\
			Dummy & -0.0149 & 0.0124 & 0.0187 & 0.0396 & 0.0005 & 0.016 \\
			& -0.92 & 0.71 & 1.01 & 1.61 & 0.04 & 0.94 \\
			Dummy$\times$RET & 0.024 & -0.0052 & 0.0056 & 0.0015 & -0.0079 & -0.0131 \\
			& 0.47 & -0.08 & 0.16 & 0.03 & -0.29 & -0.32 \\
			Dummy$\times$NEG & -0.0042 & -0.0269 & 0.0085 & -0.0035 & 0.007 & -0.0005 \\
			& -0.15 & -0.68 & 0.68 & -0.23 & 0.59 & -0.03 \\
			Constant & 0.0235 & 0.0394* & 0.0900*** & 0.1014*** & 0.1364*** & 0.1368*** \\
			& 1.42 & 2.29 & 4.12 & 4.74 & 11.45 & 10.80 \\
			Year Fixed Effects (Main) & Yes & Yes & Yes & Yes & Yes & Yes \\
			Firm Fixed Effects (Main) & Yes & Yes & Yes & Yes & Yes & Yes \\
			S.E. Clustered by States & Yes & Yes & Yes & Yes & Yes & Yes \\
			Observations & 2,220 & 1,612 & 4,241 & 3,093 & 6,134 & 4,492 \\
			Adj. R-squared & 0.4286 & 0.4169 & 0.3923 & 0.3686 & 0.3655 & 0.3543 \\
			\bottomrule
			\bottomrule
		\end{tabular}%
		\label{tab:addlabel}%
	\end{table}%
	
	\noindent This table reports results of rolling window analysis of conservatism (measured by Basu model) in three subsamples that consist of firm-year observations within one-year (-1/+1), two-years (-2/+2) and three-years (-3/+3) window before and after constituency statute enactments in all states that have adopted the law as of 2013. Column II, Column IV and Column VI show the rolling window results using only firms that headquarter in their state of incorporation. \textit{Dummy} is an indicator variable that takes 1 if this firm-year observation is recorded after law enactment, and 0 if before law enactment. The rest of variables are defined in Appendix B. EARN and RET are winsorized at 1 and 99 percentiles. All regressions control for year and firm fixed effects. Standard errors are clustered at state of incorporation level. *, **, and *** indicate significance at 10\%, 5\% and 1\% confidence level, respectively. t-statistics are reported below coefficients.
\end{landscape}

%%%%%%%%%%%%%%%%%%%%%%TABLE 5
\newgeometry{left=1cm, right=1cm, top=1.7cm, bottom=2cm}
\begin{table}[H]
	\centering
	\caption{Mechanism (1975 - 2013)}                                                 
	\begin{flushleft}
		$EARN_{i,t}=\alpha_i+\omega_t+\beta_1RET_{i,t}+\beta_2NEG_{i,t}+\beta_3RET_{i,t}\times NEG_{i,t}+POST_{i,t}\times (\beta_4+\beta_5RET_{i,t}+\beta_6NEG_{i,t}+\beta_7RET_{i,t}\times NEG_{i,t})+MEC\times(\beta_8RET_{i,t}+\beta_9NEG_{i,t}+\beta_{10}REG_{i,t}\times NEG_{i,t})+MEC\times POST_{i,t}\times (\beta_{11}+\beta_{12}RET_{i,t}+\beta_{13}NEG_{i,t}+\beta_{14}RET_{i,t}\times NEG_{i,t})+\epsilon_{i,t}$ \qquad \qquad \qquad \qquad \qquad \qquad \qquad \qquad \qquad \qquad \qquad(4)
	\end{flushleft}
	\begin{tabular}{lccc}
		\toprule
		\toprule
		Dep. Var. EARN & DCD1 & DCD2 & MA \\
		\midrule
		POST  & 0.0003 & -0.0056 & -0.0160* \\
		& 0.05  & -0.86 & -2.15 \\
		POST$\times$ RET & 0.0112 & -0.0150 & 0.0254 \\
		& 0.38  & -0.47 & 0.81 \\
		POST$\times$NEG & 0.0000 & -0.0022 & 0.0129 \\
		& 0.00  & -0.22 & 1.13 \\
		\rowcolor[rgb]{ .851,  .851,  .851} POST$\times$RET$\times$NEG & -0.0657 & 0.0316 & 0.0709 \\
		\rowcolor[rgb]{ .851,  .851,  .851}       & -1.43 & 0.61  & 1.23 \\
		MEC$\times$RET & 0.0103 & 0.0158 & 0.7419*** \\
		& 0.59  & 1.84  & 4.53 \\
		MEC$\times$NEG & -0.0039 & -0.0024 & 0.1201** \\
		& -0.36 & -0.37 & 2.98 \\
		MEC$\times$RET$\times$NEG & -0.1059* & -0.0700* & -1.5956*** \\
		& -2.46 & -2.34 & -4.21 \\
		MEC$\times$POST & -0.0100 & -0.0220* & 0.1592*** \\
		& -1.31 & -2.47 & 6.14 \\
		MEC$\times$POST$\times$RET & -0.0581* & -0.0562*** & -0.7353*** \\
		& -2.33 & -3.61 & -3.84 \\
		MEC$\times$POST$\times$NEG & 0.0023 & -0.0105 & -0.1559** \\
		& 0.18  & -0.79 & -2.80 \\
		\rowcolor[rgb]{ .851,  .851,  .851} MEC$\times$POST$\times$RET$\times$NEG & 0.2854*** & 0.1751*** & 1.1218* \\
		\rowcolor[rgb]{ .851,  .851,  .851}       & 7.74  & 4.90  & 2.65 \\
		Constant & 0.0609** & 0.0599** & 0.0742*** \\
		& 3.17  & 2.88  & 3.73 \\
		Year Fixed Effects (Main) & YES   & YES   & YES \\
		Firm Fixed Effects (Main) & YES   & YES   & YES \\
		Year Fixed Effects (Basu Coefficients) & YES   & YES   & YES \\
		State Fixed Effects (Basu Coefficients) & YES   & YES   & YES \\
		S.E. Clustered by States & YES   & YES   & YES \\
		Observations    & 23219 & 23219 & 18277 \\
		Adj. R-squared & 0.318 & 0.3334 & 0.2836 \\
		\bottomrule
		\bottomrule
	\end{tabular}%
\end{table}%
\noindent This table reports results of Equation (4): earnings on baseline Basu factors (i.e., NEG, RET, NEG $\times$ RET) and their interactions with POST dummy and three mechanism variables (DCD1, DCD2 and MA). All variables are defined in Appendix B. EARN and RET are winsorized at 1 and 99 percentiles. All regressions control for main firm and year fixed effects and Basu coefficient fixed effects. Standard errors are clustered at state of incorporation level. The Basu coefficient fixed effects absorb slope coefficients in baseline Basu model so they are omitted.  *, **, and *** indicate significance at 10\%, 5\% and 1\% confidence level respectively. T-statistics are reported below coefficients.

\restoregeometry

%%%%%%%%%%%%%%%%%%%TABLE 6
\begin{table}[H]
	\centering
	\caption{KLD Indexes (1995 - 2013)}
	
	Panel A. KLD Summary Statistics
	\begin{tabular}{lrrrrrrrrrr}
		\toprule
		\toprule
		& \multicolumn{1}{l}{N} & \multicolumn{1}{l}{mean} & \multicolumn{1}{l}{median} & \multicolumn{1}{l}{std. dev.} & \multicolumn{1}{l}{max} & \multicolumn{1}{l}{min} & \multicolumn{1}{l}{p1} & \multicolumn{1}{l}{p25} & \multicolumn{1}{l}{p75} & \multicolumn{1}{l}{p99} \\
		\midrule
		KLD\_STR & 12341 & 0.476 & 0     & 1.060 & 12    & 0     & 0     & 0     & 1     & 5 \\
		KLD\_CON & 13922 & 1.011 & 1     & 1.518 & 13    & 0     & 0     & 0     & 1     & 7 \\
		KLD\_TOTAL & 12341 & -0.515 & 0     & 1.403 & 8     & -9    & -5    & -1    & 0     & 3 \\
		Enviroment & 15800 & 0.137 & 0     & 0.427 & 4     & 0     & 0     & 0     & 0     & 2 \\
		Community & 15624 & 0.147 & 0     & 0.476 & 5     & 0     & 0     & 0     & 0     & 2 \\
		Product & 15624 & 0.071 & 0     & 0.272 & 3     & 0     & 0     & 0     & 0     & 1 \\
		Emloyee relation & 12341 & 0.241 & 0     & 0.571 & 5     & 0     & 0     & 0     & 0     & 2 \\
		Human rights & 12986 & 0.003 & 0     & 0.054 & 1     & 0     & 0     & 0     & 0     & 0 \\
		\bottomrule
		\bottomrule
	\end{tabular}%
	
	\bigbreak
	Panel B. KLD Indexes Correlation Matrix
	\begin{tabular}{lrrrrrrrr}
		\toprule
		\toprule
		& KLD\_STR & KLD\_CON & KLD\_TOTAL & Env. & Com. & Prod. & Emp. & Hum.\\
		\midrule
		KLD\_STR & 1.000 &       &       &       &       &       &       &  \\
		KLD\_CON & 0.425 & 1.000 &       &       &       &       &       &  \\
		KLD\_TOTAL & 0.309 & -0.730 & 1.000 &       &       &       &       &  \\
		Enviroment & 0.655 & 0.331 & 0.122 & 1.000 &       &       &       &  \\
		Community & 0.662 & 0.305 & 0.174 & 0.264 & 1.000 &       &       &  \\
		Product & 0.500 & 0.116 & 0.235 & 0.204 & 0.149 & 1.000 &       &  \\
		Employee relation & 0.801 & 0.307 & 0.282 & 0.306 & 0.287 & 0.256 & 1.000 &  \\
		Human rights & 0.145 & 0.043 & 0.062 & 0.023 & 0.117 & 0.031 & 0.064 & 1.000 \\
		\bottomrule
		\bottomrule
	\end{tabular}%
\end{table}%
\noindent This table reports summary statistics (Panel A) and correlation matrix (Panel B) of KLD indexes from 1995 to 2013. All indexes are defined in Appendix C.

%%%%%%%%%%%%%%%%%%TABLE 7
\newgeometry{left=1cm, right=1cm, top=2cm, bottom=2cm}
\begin{landscape}
	\begin{table}[H]
		\centering
		\caption{Association between KLD Indexes and Conservatism , Measured by C\textunderscore Score (1995 - 2013)}
		
		$Conservatism_{i,t}=a_i+b_t+c\times KLD\textunderscore INDEX+d\times X_{i,t}+\epsilon_{i,t}$ \qquad (8)
		
		\begin{tabular}{lccccccc}
			\toprule
			\toprule
			Dependent Variable: C\_Score & I     & II    & III   & IV    & V     & VI    & VII \\
			\midrule
			KLD\_STR & \cellcolor[rgb]{ .906,  .902,  .902}0.0030*** & \cellcolor[rgb]{ .906,  .902,  .902} & \cellcolor[rgb]{ .906,  .902,  .902} & \cellcolor[rgb]{ .906,  .902,  .902} & \cellcolor[rgb]{ .906,  .902,  .902} & \cellcolor[rgb]{ .906,  .902,  .902} & \cellcolor[rgb]{ .906,  .902,  .902}              \\
			& \cellcolor[rgb]{ .906,  .902,  .902}4.95 & \cellcolor[rgb]{ .906,  .902,  .902} & \cellcolor[rgb]{ .906,  .902,  .902} & \cellcolor[rgb]{ .906,  .902,  .902} & \cellcolor[rgb]{ .906,  .902,  .902} & \cellcolor[rgb]{ .906,  .902,  .902} & \cellcolor[rgb]{ .906,  .902,  .902} \\
			KLD\textunderscore TOTAL   & \cellcolor[rgb]{ .906,  .902,  .902} & \cellcolor[rgb]{ .906,  .902,  .902}-0.0011* & \cellcolor[rgb]{ .906,  .902,  .902} & \cellcolor[rgb]{ .906,  .902,  .902} & \cellcolor[rgb]{ .906,  .902,  .902} & \cellcolor[rgb]{ .906,  .902,  .902} & \cellcolor[rgb]{ .906,  .902,  .902} \\
			& \cellcolor[rgb]{ .906,  .902,  .902} & \cellcolor[rgb]{ .906,  .902,  .902}-2.07 & \cellcolor[rgb]{ .906,  .902,  .902} & \cellcolor[rgb]{ .906,  .902,  .902} & \cellcolor[rgb]{ .906,  .902,  .902} & \cellcolor[rgb]{ .906,  .902,  .902} & \cellcolor[rgb]{ .906,  .902,  .902} \\
			Environment & \cellcolor[rgb]{ .906,  .902,  .902} & \cellcolor[rgb]{ .906,  .902,  .902} & \cellcolor[rgb]{ .906,  .902,  .902}-0.0019 & \cellcolor[rgb]{ .906,  .902,  .902} & \cellcolor[rgb]{ .906,  .902,  .902} & \cellcolor[rgb]{ .906,  .902,  .902} & \cellcolor[rgb]{ .906,  .902,  .902} \\
			& \cellcolor[rgb]{ .906,  .902,  .902} & \cellcolor[rgb]{ .906,  .902,  .902} & \cellcolor[rgb]{ .906,  .902,  .902}-1.93 & \cellcolor[rgb]{ .906,  .902,  .902} & \cellcolor[rgb]{ .906,  .902,  .902} & \cellcolor[rgb]{ .906,  .902,  .902} & \cellcolor[rgb]{ .906,  .902,  .902} \\
			Community & \cellcolor[rgb]{ .906,  .902,  .902} & \cellcolor[rgb]{ .906,  .902,  .902} & \cellcolor[rgb]{ .906,  .902,  .902} & \cellcolor[rgb]{ .906,  .902,  .902}-0.0020* & \cellcolor[rgb]{ .906,  .902,  .902} & \cellcolor[rgb]{ .906,  .902,  .902} & \cellcolor[rgb]{ .906,  .902,  .902} \\
			& \cellcolor[rgb]{ .906,  .902,  .902} & \cellcolor[rgb]{ .906,  .902,  .902} & \cellcolor[rgb]{ .906,  .902,  .902} & \cellcolor[rgb]{ .906,  .902,  .902}-2.45 & \cellcolor[rgb]{ .906,  .902,  .902} & \cellcolor[rgb]{ .906,  .902,  .902} & \cellcolor[rgb]{ .906,  .902,  .902} \\
			Product & \cellcolor[rgb]{ .906,  .902,  .902} & \cellcolor[rgb]{ .906,  .902,  .902} & \cellcolor[rgb]{ .906,  .902,  .902} & \cellcolor[rgb]{ .906,  .902,  .902} & \cellcolor[rgb]{ .906,  .902,  .902}0.0060* & \cellcolor[rgb]{ .906,  .902,  .902} & \cellcolor[rgb]{ .906,  .902,  .902} \\
			& \cellcolor[rgb]{ .906,  .902,  .902} & \cellcolor[rgb]{ .906,  .902,  .902} & \cellcolor[rgb]{ .906,  .902,  .902} & \cellcolor[rgb]{ .906,  .902,  .902} & \cellcolor[rgb]{ .906,  .902,  .902}2.42 & \cellcolor[rgb]{ .906,  .902,  .902} & \cellcolor[rgb]{ .906,  .902,  .902} \\
			Employee Relation & \cellcolor[rgb]{ .906,  .902,  .902} & \cellcolor[rgb]{ .906,  .902,  .902} & \cellcolor[rgb]{ .906,  .902,  .902} & \cellcolor[rgb]{ .906,  .902,  .902} & \cellcolor[rgb]{ .906,  .902,  .902} & \cellcolor[rgb]{ .906,  .902,  .902}0.0028** & \cellcolor[rgb]{ .906,  .902,  .902} \\
			& \cellcolor[rgb]{ .906,  .902,  .902} & \cellcolor[rgb]{ .906,  .902,  .902} & \cellcolor[rgb]{ .906,  .902,  .902} & \cellcolor[rgb]{ .906,  .902,  .902} & \cellcolor[rgb]{ .906,  .902,  .902} & \cellcolor[rgb]{ .906,  .902,  .902}3.47 & \cellcolor[rgb]{ .906,  .902,  .902} \\
			Human Rights & \cellcolor[rgb]{ .906,  .902,  .902} & \cellcolor[rgb]{ .906,  .902,  .902} & \cellcolor[rgb]{ .906,  .902,  .902} & \cellcolor[rgb]{ .906,  .902,  .902} & \cellcolor[rgb]{ .906,  .902,  .902} & \cellcolor[rgb]{ .906,  .902,  .902} & \cellcolor[rgb]{ .906,  .902,  .902}-0.0059 \\
			& \cellcolor[rgb]{ .906,  .902,  .902} & \cellcolor[rgb]{ .906,  .902,  .902} & \cellcolor[rgb]{ .906,  .902,  .902} & \cellcolor[rgb]{ .906,  .902,  .902} & \cellcolor[rgb]{ .906,  .902,  .902} & \cellcolor[rgb]{ .906,  .902,  .902} & \cellcolor[rgb]{ .906,  .902,  .902}-0.80 \\
			SIZE  & -0.0094*** & -0.0093*** & -0.0114*** & -0.0108*** & -0.0108*** & -0.0095*** & -0.0102*** \\
			& -8.41 & -8.11 & -9.34 & -8.64 & -8.74 & -8.18 & -8.77 \\
			MTB   & 0.0052*** & 0.0052*** & 0.0055*** & 0.0053*** & 0.0054*** & 0.0052*** & 0.0058*** \\
			& 26.27 & 26.17 & 15.15 & 18.34 & 18.19 & 26.28 & 28.29 \\
			LEV   & 0.061*** & 0.061*** & 0.0576*** & 0.0574*** & 0.0574*** & 0.0610*** & 0.0580*** \\
			& 24.19 & 23.88 & 26.32 & 26.55 & 26.48 & 23.99 & 22.85 \\
			Constant & 0.2287*** & 0.2287*** & 0.0176 & 0.0148 & 0.0136 & 0.2295*** & 0.1523*** \\
			& 30.31 & 29.85 & 1.96  & 1.57  & 1.43  & 29.76 & 22.15 \\
			Year fixed effects & YES   & YES   & YES   & YES   & YES   & YES   & YES \\
			Firm Fixed Effects & YES   & YES   & YES   & YES   & YES   & YES   & YES \\
			S.E. Clustered by States & YES   & YES   & YES   & YES   & YES   & YES   & YES \\
			Observations    & 10483 & 10483 & 13920 & 13742 & 13742 & 10483 & 11087 \\
			Adj. R-square & 0.7778 & 0.7776 & 0.778 & 0.7752 & 0.7753 & 0.7776 & 0.7708 \\
			\bottomrule
			\bottomrule
		\end{tabular}%
		\label{tab:addlabel}%
	\end{table}%
	
	\noindent This table reports results of Equation (8): OLS regression model of conservatism (measured by C\textunderscore Score) on KLD indexes. All KLD indexes are defined in Appendix c. All regressions control for firm and year fixed effects. Standard errors are clustered at state of incorporation level. *, **, and *** indicate significance at 10\%, 5\% and 1\% confidence level respectively. T-statistics are reported below coefficients.
	
\end{landscape}
\restoregeometry

%%%%%%%%%%%TABLE 8
\begin{table}[H]
	\centering
	\caption{DiD: Effect  of Constituency Statute Enactments on C\textunderscore Score, Three Periods}
	$Conservatism_{i,t}=a_i+b_t+c\times POST_{i,t}+d\times X_{i,t}+\epsilon_{i,t}$ \qquad (9)
	\begin{tabular}{lccc}
		\toprule
		\toprule
		Dep. Var. C\_Score & 1975 - 2013 & 1984 - 1992 & 1985 - 1990 \\
		& I & II & III \\
		\midrule
		\rowcolor[rgb]{ .906,  .902,  .902} POST  & 0.0046 & 0.0073* & 0.0112* \\
		\rowcolor[rgb]{ .906,  .902,  .902}       & 1.56  & 2.17  & 2.51 \\
		SIZE  & -0.0150*** & -0.0355*** & -0.0441*** \\
		& -8.27 & -12.94 & -8.83 \\
		MTB   & -0.0018 & -0.0051*** & -0.0044* \\
		& -1.64 & -5.83 & -2.73 \\
		LEV   & 0.0633*** & 0.0329*** & 0.0218*** \\
		& 23.07 & 10.45 & 3.99 \\
		Constant & 0.0824*** & 0.1695*** & 0.2749*** \\
		& 8.98  & 12.68 & 11.41 \\
		Year Fixed Effects & YES   & YES   & YES \\
		Firm Fixed Effects & YES   & YES   & YES \\
		S.E. Clustered by States & YES   & YES   & YES \\
		Observations  & 22570 & 6965  & 4644 \\
		Adj. R-square & 0.5016 & 0.5482 & 0.5194 \\
		\bottomrule
		\bottomrule
	\end{tabular}%
\end{table}%

\noindent This table reports DiD regression (Equation 9) results in three sample periods: 1975 - 2013 (Column I), 1984 - 1992 (Column II) and 1985 - 1990 (Column III). All variables are defined in Appendix B. SIZE, MTB and LEV are winsorized at 1 and 99 percentiles. All regressions control for firm and year fixed effects. Standard errors are clustered at state of incorporation level. *, **, and *** indicate significance at 10\%, 5\% and 1\% confidence level respectively. T-statistics are reported below coefficients.

%%%%%%%%%%%%TABLE 9
\begin{table}[H]
	\centering
	\caption{DiD: Effect of Constituency Statute Enactments on C\textunderscore Score, Rolling Windows}
	$Conservatism_{i,t}=a_i+b_t+c\times Dummy_{i,t}+d\times X_{i,t}+\epsilon_{i,t}$
	\begin{tabular}{lccc}
		\toprule
		\toprule
		Dep. Var. C\_SCORE & I & II & III \\
		\multicolumn{1}{r}{} & -1/+1 & -2/+2 & -3/+3 \\
		\midrule
		\rowcolor[rgb]{ .906,  .902,  .902} Dummy & 0.0223*** & 0.011 & 0.0083 \\
		\rowcolor[rgb]{ .906,  .902,  .902} & 5.89  & 1.09  & 1.36 \\
		SIZE  & -0.0387** & -0.0303***& -0.0318*** \\
		& -3.03 & -5.74 & -6.92 \\
		MTB   & -0.0032 & -0.0049** & -0.0041** \\
		& -1.02 & -3.26 & -3.56 \\
		LEV   & 0.0164 & 0.0319** & 0.0341*** \\
		& 0.74  & 2.92  & 5.13 \\
		Constant & 0.2184*** & 0.2130*** & 0.1846*** \\
		& 3.80  & 8.03  & 8.11 \\
		Year Fixed Effects & Yes & Yes & Yes \\
		Firm Fixed Effects & Yes & Yes & Yes \\
		S.E. Clustered by States & Yes & Yes & Yes \\
		Observations & 2,144 & 4,105 & 5,941 \\
		Adj. R-squared & 0.6858 & 0.7176 & 0.6896 \\
		\bottomrule
		\bottomrule
	\end{tabular}%
\end{table}%
\noindent This table reports results of rolling window analysis of conservatism (measured by C\textunderscore Score) in three subsamples that consist of firm-year observations within one-year (-1/+1), two-years (-2/+2) and three-years (-3/+3) window before and after constituency statute enactments in all states that have adopted the law as of 2013. \textit{Dummy} is an indicator variable that takes 1 if this firm-year observation is recorded after law enactment, and 0 if before law enactment. The rest of variables are defined in Appendix B. SIZE, MTB and LEV are winsorized at 1 and 99 percentiles. All regressions control for year and firm fixed effects. Standard errors are clustered at state of incorporation level. *, **, and *** indicate significance at 10\%, 5\% and 1\% confidence level, respectively. t-statistics are reported below coefficients.
\end{document}

